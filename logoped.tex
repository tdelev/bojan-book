\documentclass[a5paper,12pt]{article}

% XeLaTeX setup for Cyrillic
\usepackage{fontspec}
\usepackage{polyglossia}
\setmainlanguage{macedonian}
\setotherlanguage{english}

% Select a font with good Cyrillic support
\setmainfont{Noto Serif}
\newfontfamily\cyrillicfont{Noto Serif}
% Alternatives: , Noto Serif, FreeSerif
\usepackage{lmodern}            % Latin Modern fonts
\usepackage{geometry}           % Page geometry
\usepackage{graphicx}           % For including images
\usepackage{xcolor}             % For colored text
\usepackage{tcolorbox}          % For creating colored boxes
\usepackage{enumitem}           % For customizing lists
\usepackage{titlesec}           % For customizing section titles
\usepackage{fancyhdr}           % For headers and footers
\usepackage{multirow}           % For tables
\usepackage{multicol}
\usepackage{wrapfig}            % For wrapping text around figures
\usepackage{hyperref}           % For hyperlinks

% Page geometry for booklet format
\geometry{
  a5paper,
  top=1.5cm,
  bottom=1.5cm,
  left=1.5cm,
  right=1.5cm,
  footskip=1cm
}

% Define colors for the document
\definecolor{titlecolor}{RGB}{0,102,204}
\definecolor{boxcolor}{RGB}{230,242,255}
\definecolor{accentcolor}{RGB}{255,153,0}
\definecolor{warmred}{RGB}{240,37,37}
\definecolor{warmredbg}{RGB}{252,233,233}

% Customize section titles
\titleformat{\section}
  {\color{titlecolor}\Large\bfseries}
  {\thesection.}{0.5em}{}

\titleformat{\subsection}
  {\color{titlecolor}\large\bfseries}
  {\thesubsection.}{0.5em}{}

% Custom header/footer
\pagestyle{fancy}
\fancyhf{}
\renewcommand{\headrulewidth}{0.4pt}
\renewcommand{\footrulewidth}{0.4pt}
\fancyhead[C]{\textcolor{titlecolor}{Логопед}} 
\fancyfoot[C]{\thepage}

% Custom environments
\newenvironment{activity}[1]{%
  \begin{tcolorbox}[colback=boxcolor,colframe=titlecolor,title={\textbf{#1}},fonttitle=\bfseries]
}{%
  \end{tcolorbox}
}

\newenvironment{instruction}{%
  \begin{tcolorbox}[colback=white,colframe=accentcolor,title={\textbf{Инструкции}},fonttitle=\bfseries] 
}{%
  \end{tcolorbox}
}

% Custom environments
\newenvironment{song}{%
  \begin{tcolorbox}[colback=warmredbg,colframe=warmred,title={\textbf{Песничка}},fonttitle=\bfseries]
}{%
  \end{tcolorbox}
}
% Document begins
\begin{document}

% Title page
\begin{titlepage}
\centering
{\Huge\textcolor{titlecolor}{\textbf{Логопед}}\par} 
\vspace{1cm}
{\Large\textcolor{accentcolor}{Бојан}\par} 
\vspace{2cm}

% \includegraphics[width=0.6\textwidth]{example-image} % Replace with your logo/image

\vspace{2cm}
{\large Изработено:\par} 
{\large\textit{Томче Делев}\par}
\vspace{1cm}
{\large\today\par}
\end{titlepage}

% Table of contents
\tableofcontents
\newpage

\section{06.11.2024} 

\begin{instruction}
Држете ги стистнати забчињата, јазичето да не „ѕирка“ меѓу запчиња.
\end{instruction}

\noindent % Ensures no paragraph indentation
\begin{minipage}[t]{0.19\textwidth}
  са-са\\
  са-се\\
  са-си\\
  са-со\\
  са-су
\end{minipage}
\hfill 
\begin{minipage}[t]{0.19\textwidth}
  се-са\\
  се-се\\
  се-си\\
  се-со\\
  се-су
\end{minipage}
\hfill % Horizontal fill
\begin{minipage}[t]{0.19\textwidth}
  си-са\\
  си-се\\
  си-си\\
  си-со\\
  си-су
\end{minipage}
\hfill % Horizontal fill
\begin{minipage}[t]{0.19\textwidth}
  со-са\\
  со-се\\
  со-си\\
  со-со\\
  со-су
\end{minipage}
\hfill % Horizontal fill - pushes columns apart evenly
\begin{minipage}[t]{0.19\textwidth}
  су-са\\
  су-се\\
  су-си\\
  су-со\\
  су-су
\end{minipage}

\vspace{1cm}

\noindent 
\begin{minipage}[t]{0.19\textwidth}
  а-са\\
  а-се\\
  а-си\\
  а-со\\
  а-су
\end{minipage}
\hfill 
\begin{minipage}[t]{0.19\textwidth}
  е-са\\
  е-се\\
  е-си\\
  е-со\\
  е-су
\end{minipage}
\hfill % Horizontal fill
\begin{minipage}[t]{0.19\textwidth}
  и-са\\
  и-се\\
  и-си\\
  и-со\\
  и-су
\end{minipage}
\hfill % Horizontal fill
\begin{minipage}[t]{0.19\textwidth}
  о-са\\
  о-се\\
  о-си\\
  о-со\\
  о-су
\end{minipage}
\hfill % Horizontal fill - pushes columns apart evenly
\begin{minipage}[t]{0.19\textwidth}
  у-са\\
  у-се\\
  у-си\\
  у-со\\
  у-су
\end{minipage}

\section{08.11.2024}

\begin{activity}{Зборчиња} % "Exercise 'Smile'" in Russian
\begin{multicols}{4}
сака\\ сапун\\ сарма\\ саат\\ салата\\ сала\\ само\\ салеп\\ сако\\ сања\\ Сани\\ Саво\\ саќе\\ Сандра\\ Самка\\ село\\ седум\\ сече\\ семка\\ семки\\ сено\\ секоја\\ секој\\ сенка\\ сила\\
сино\\ Симо\\ сиво\\ сика\\ сипа\\ сирење\\ сито\\ Сирма\\ соба\\ сок\\ сорта\\ сонце\\ сомун\\ Солун\\ сон\\ сони\\ сонува\\ Сопиште\\ сопка\\ сопна\\ сокна\\ сонце\\ супа\\ сутура\\ сука\\
сукало\\ сурла\\ сума\\ суди\\ судија\\ смоки\\ смел\\ смее\\ смело\\ спои\\ спојувалка\\ скока\\ скок\\ Скопје\\ скали\\ снешко\\ снег\\ стои\\ сто\\ стар
\end{multicols}

\end{activity}


\section{13.11.2024} 

\begin{activity}{Зборчиња}
\begin{multicols}{4}
маса\\ каса\\ реси\\ носи\\ Веса\\ меси\\ Тасе\\ Таса\\ месо\\ високо\\ досадно\\ посипа\\ масичка\\ коса\\ кеса\\ висина\\ коси\\ песок\\ расеа\\
масира\\ пасира\\ Васо\\ Василка\\ косилка\\ носилка\\ поседи\\ боса\\ роса\\ носе\\ Деси\\ мисирка\\ наседна\\
коска\\ маска\\ Васка\\ Роска\\ тесно\\ десно\\ писмо\\ место\\ виски\\ масно\\ тесто\\ песто\\ костени\\ Коста\\ пости\\ вести
\end{multicols}
\end{activity}

\section{13.11.2024} 

\begin{activity}{Зборчиња}
\begin{multicols}{4}
чипс\\ вис\\ мис\\ кокос\\ ананас\\ пипс\\ кос\\ мирис\\ полис\\ сестра\\ спаси\\ Спасе\\ свест\\ месесто\\ состојки\\ состои\\ смеса\\
смисол\\ смести\\ сос\\ сусам\\ свесен\\ спасен\\ спуст\\ снесе\\ сласт\\ висост\\ состави\\ составува\\ сосови\\ сиресто\\ спласната\\ спасува
\end{multicols}
\end{activity}

\section{22.11.2024} 
\begin{itemize}
  \item Симе носи ананас.
  \item Во среда ќе одам во Скопје.
  \item Сања носи сукња.
  \item Се сопнав на скалите.
  \item Ми се насмевна Саше.
  \item Списокот е целосно пополнет.
  \item Ми се грицкаат смоки.
  \item Ќе пијам сок од праска.
  \item Седев и се замислив.
  \item За ручек ќе јадам супа.
  \item Сакам да скокам.
  \item Ќе се запишам на спорт.
  \item Ми се спие.
  \item Соња има син автомобил.
  \item Ја скратив косата.
  \item Веса коси со косилка.
  \item Стево стави сол во салатата.
  \item Многу е кисел сокот.
  \item Се скрши јајцето и излезе диносауросот.
\end{itemize}


\section{27.11.2024}
\noindent 
\begin{minipage}[t]{0.19\textwidth}
  за-за\\
  за-зе\\
  за-зи\\
  за-зо\\
  за-зу
\end{minipage}
\hfill 
\begin{minipage}[t]{0.19\textwidth}
  зе-за\\
  зе-зе\\
  зе-зи\\
  зе-зо\\
  зе-зу
\end{minipage}
\hfill % Horizontal fill
\begin{minipage}[t]{0.19\textwidth}
  зи-за\\
  зи-зе\\
  зи-зи\\
  зи-зо\\
  зи-зу
\end{minipage}
\hfill % Horizontal fill
\begin{minipage}[t]{0.19\textwidth}
  зо-за\\
  зо-зе\\
  зо-зи\\
  зо-зо\\
  зо-зу
\end{minipage}
\hfill % Horizontal fill - pushes columns apart evenly
\begin{minipage}[t]{0.19\textwidth}
  зу-за\\
  зу-зе\\
  зу-зи\\
  зу-зо\\
  зу-зу
\end{minipage}

\vspace{1cm}

\begin{activity}{Зборчиња}
\begin{multicols}{4}
заб\\ заби\\ запче\\ запна\\ закон\\ замка\\ завчера\\ заспа\\ затка\\ земја\\ зема\\ зелено\\ зелка\\ зебра\\ зеза\\ зима\\
зина\\ зимско\\ Зоки\\ зошто\\ зона\\ Зока\\ знаме\\ значка\\ змија\\ зајак\\ злато\\ знае\\ змеј\\ закон\\ звук\\ звучник\\ златно\\ златко\\ збор\\ зборува
\end{multicols}
\end{activity}


\section{29.11.2024} 
\begin{activity}{Зборчиња}
\begin{multicols}{4}
боза\\ коза\\ нозе\\ Розе\\ Роза\\ вози\\ мезе\\ мези\\ лоза\\ база\\ теза\\ виза\\ низа\\ Пиза\\ казан\\ базен\\ возен\\ доза\\ маза\\ лазе\\ казна\\ базна
вазна\\ дозна\\ познат\\ бозле\\ Козле
\end{multicols}
\end{activity}

\begin{instruction}
  Бидејќи во овие зборови гласот \textbf{з} го обезвучува, кога го читате, одвојте го \textbf{з} од остнатите гласови. Пр. \emph{ма-ззз-на}
\end{instruction}

\section{04.12.2024} 
\begin{activity}{Зборчиња}
\begin{multicols}{4}
доказ\\ патоказ\\ млаз\\ пегаз\\ папаз\\ вез
\end{multicols}
\end{activity}
\begin{itemize}
  \item Зелката е кисела.
  \item Го следам патоказот.
  \item Лазе ја скрши вазната.
  \item Ми се скрши срцето.
  \item Роза вози брзо.
  \item Розе пие боза. 
  \item Базенот е зелен.
  \item Зоки фати змија.
  \item Зајакот трча брзо.
  \item Змијата е златна.
\end{itemize}

\section{11.12.2024} 
\begin{instruction}
Гласот Ц се изговара слично како С и З, со стиснати запчиња и јазиче внатре.
\end{instruction}

\noindent 
\begin{center}
  ца\\
  це\\
  ци\\
  цо\\
  цу
\end{center}
\begin{minipage}[t]{0.19\textwidth}
  аца\\
  аце\\
  аци\\
  ацо\\
  ацу
\end{minipage}
\hfill
\begin{minipage}[t]{0.19\textwidth}
  еца\\
  еце\\
  еци\\
  ецо\\
  ецу
\end{minipage}
\hfill
\begin{minipage}[t]{0.19\textwidth}
  ица\\
  ице\\
  ици\\
  ицо\\
  ицу
\end{minipage}
\hfill 
\begin{minipage}[t]{0.19\textwidth}
  оца\\
  оце\\
  оци\\
  оцо\\
  оцу
\end{minipage}
\hfill 
\begin{minipage}[t]{0.19\textwidth}
  уца\\
  уце\\
  уци\\
  уцо\\
  уцу
\end{minipage}

\vspace{1cm}
\begin{minipage}[t]{0.18\textwidth}
  цаца\\
  цаце\\
  цаци\\
  цацо\\
  цацу
\end{minipage}
\hfill
\begin{minipage}[t]{0.18\textwidth}
  цеца\\
  цеце\\
  цеци\\
  цецо\\
  цецу
\end{minipage}
\hfill
\begin{minipage}[t]{0.18\textwidth}
  цица\\
  цице\\
  цици\\
  цицо\\
  цицу
\end{minipage}
\hfill 
\begin{minipage}[t]{0.18\textwidth}
  цоца\\
  цоце\\
  цоци\\
  цоцо\\
  цоцу
\end{minipage}
\hfill 
\begin{minipage}[t]{0.18\textwidth}
  цуца\\
  цуце\\
  цуци\\
  цуцо\\
  цуцу
\end{minipage}

\begin{activity}{Зборчиња}
\begin{multicols}{4}
цврче\\ цапна\\ цака\\ Цане\\ Цако\\ цело\\ целофан\\ целина\\ Цеко\\ цивка\\ цигари\\ Цима\\ Цоне\\ цокле\\ Цобе\\ цути\\ Цуне\\ Цуки
\end{multicols}
\end{activity}

\section{13.12.2024}
\begin{activity}{Зборчиња}
\begin{multicols}{4}
Маца\\ каца\\ Коце\\ Поце\\ боца\\ веце\\ лице\\ Мице\\ Коце\\ баце\\ Ница\\ пица\\ жица\\ редица\\ меница\\ меденица\\ ластовица\\ корица\\
седмица\\ меца\\ деца\\ коленица\\ паница\\ баница\\ Анкица\\ Радица\\ Надица\\ Даница\\ рамница\\ Тоци\\ боци\\ Деница\\ Дениција
\end{multicols}
\end{activity}


\section{27.12.2024}
\begin{activity}{Зборчиња}
\begin{multicols}{4}
конец\\ боц\\ ланец\\ котелец\\ палец\\ мац\\ патец\\ виц\\ ноктец\\ вранец\\ врабец\\ лонец\\ ранец\\ колец\\ венец\\ пустец\\ данец\\ Цацко\\ цуцла\\ Цеце\\ цица\\ Цацо
\end{multicols}
\end{activity}
\begin{itemize}
  \item Бебето цуца цуцла.
  \item Маца го скина цвеќето.
  \item Мице го повреди палецот.
  \item Раца направи баница.
  \item Децата јадат мекици.
  \item Во лонецот се вари цвекло.
  \item На корицата се Каца и Коце.
  \item Во ранецот имам црна блуза.
  \item Цигарите се во целофан.
  \item Цвеќето цути на пролет.
  \item Глувчето Мице цивка. 
  \item Децата видоа ластовица. 
  \item Анкица прави пица.
  \item Лицето ми е црвено.
  \item И цвеклото е црвено.
  \item Во ваницата има сиренце.
  \item Цобе се боцна на жица.
  \item Цеце отиде во Ница.
\end{itemize}

\section{11.12.2024} 
\begin{instruction}
Во текот на секојдневието корегирајте го неправилниот изговор на гласовите кои до сега се работени, за да стане свесен за нивната правилна употреба.
\end{instruction}
\begin{itemize}
  \item На троседот седи зелена гасеница.
  \item Слонот цапна во големиот базен.
  \item Зајачето скока високо и трча брзо.
  \item Грозјето е зелено и слатко.
  \item Зорица ми испржи јајца.
  \item Многу сакам да јадам краставички.
  \item Лавот се исплаши од глувчето.
  \item Бебето цуца цуцла.
  \item Златко носи црвена маица.
  \item Во теглата има шарени цевки.
  \item Ѕвончето ѕвони гласно.
  \item Ѕидарот ѕида ѕид.
  \item Ѕвездата свети на неботo.
\end{itemize}

\section{15.01.2025}
\begin{song}
  Ќе наберам цвеќе најубаво на свет\\
  и ќе ја гушнам мама послатка од мед.\\

  Ќе ја стегнам милно, и ќе ја гушнам силно.\\
  Сонце си што грее Птица, си што пее.\\

  Најдобра на свет и од мајќи пет!
\end{song}

\section{17.01.2025}
Вежби за автоматизација на гласовите С, З и Ц со опис на слики.

\section{22.01.2025}
\begin{itemize}
  \item Ѕвончето ѕвони гласно.
  \item Ѕзвездата свети силно.
  \item Ѕидарот го соѕида ѕидот.
  \item Ѕвонко заѕвони на ѕвончето.
  \item Ѕвезда сука баница.
  \item Ѕидот е сина боја.
  \item Цане виде ѕуница.
  \item Цвеќето на масата е звездесто.
  \item Дедо Мраз има ѕвонче во раката.
  \item На Саше му ѕвони телефонот.
  \item Цвета направи ѕвезда на гимнастика.
\end{itemize}

\section{24.01.2025}
Вежби за автоматизација на гласовите С, З, Ц и Ѕ со опис на слики и сликовници.

\section{29.01.2025}
\begin{instruction}
  \begin{enumerate}
    \item Наизменично ќе се насмевне широко, па ќе ги собере усничките.
    \item Во насмевка кажува долго \textbf{СССС}, а со собрани усни долго \textbf{ШШШШ} наизменично во низа.
  \end{enumerate}
\end{instruction}

\section{31.01.2025}
\noindent 
\begin{minipage}[t]{0.18\textwidth}
  шаша\\
  шаше\\
  шаши\\
  шашо\\
  шашу
\end{minipage}
\begin{minipage}[t]{0.18\textwidth}
  шеша\\
  шеше\\
  шеши\\
  шешо\\
  шешу
\end{minipage}
\hfill
\begin{minipage}[t]{0.18\textwidth}
  шиша\\
  шише\\
  шиши\\
  шишо\\
  шишу
\end{minipage}
\hfill
\begin{minipage}[t]{0.18\textwidth}
  шоша\\
  шоше\\
  шоши\\
  шошо\\
  шошу
\end{minipage}
\hfill 
\begin{minipage}[t]{0.18\textwidth}
  шуша\\
  шуше\\
  шуши\\
  шушо\\
  шушу
\end{minipage}

\vspace{1cm}

\noindent 
\begin{minipage}[t]{0.18\textwidth}
  aша\\
  aше\\
  aши\\
  aшо\\
  aшу
\end{minipage}
\begin{minipage}[t]{0.18\textwidth}
  еша\\
  еше\\
  еши\\
  ешо\\
  ешу
\end{minipage}
\hfill
\begin{minipage}[t]{0.18\textwidth}
  иша\\
  ише\\
  иши\\
  ишо\\
  ишу
\end{minipage}
\hfill
\begin{minipage}[t]{0.18\textwidth}
  оша\\
  оше\\
  оши\\
  ошо\\
  ошу
\end{minipage}
\hfill 
\begin{minipage}[t]{0.18\textwidth}
  уша\\
  уше\\
  уши\\
  ушо\\
  ушу
\end{minipage}

\section{05.02.2025}
\begin{activity}{Зборчиња}
\begin{multicols}{4}
шака\\ шара\\ Шаде\\ шамија\\ шав\\ шахта\\ шах\\ шатор\\ шатка\\ Шана\\ шамар\\ шам\\ шета\\ шеќер\\ шема\\ Шенка\\ шепа\\ шепот\\ шепоти\\ шепка\\ шило\\ шини\\ шивач\\
шие\\ шипка\\ шипиники\\ шок\\ шокира\\ Шон\\ шубара\\ шут\\ шутира\\ шули\\ шума\\ шумски\\ шунка\\ Шутка\\ шуто\\ шуплина\\ шупа\\ шупло\\ шуливо\\ шука
\end{multicols}
\end{activity}

\section{07.02.2025}
\begin{activity}{Зборчиња}
\begin{multicols}{4}
меше\\ дише\\ каша\\ вишо\\ Дишо\\ Моша\\ ваше\\ седеше\\ молеше\\ Нешо\\ пуши\\ пешак\\ Тоше\\ брише\\ избриша\\ велеше\\ меша\\ решетка\\ решето\\ реши\\ Рашо\\ буши
избуши\\ вадеше\\ ладеше\\ пееше\\ прошета\\ мушичка\\ пишува\\ болеше\\ плаши\\ глушец\\ прашина\\ нишесте\\ рашири\\ пошироко\\ пошира\\ тушира\\ кокошарник
\end{multicols}
\end{activity}

\section{12.02.2025}
\begin{activity}{Зборчиња}
\begin{multicols}{4}
кош\\ пеш\\ дадеш\\ веш\\ мош\\ полниш\\ Радовиш\\ Ниш\\ вениш\\ каниш\\ боиш\\ бош\\ надеж\\ шишарка\\ шашаво\\ шешир\\ шушка\\ шушти\\ шушлер\\ шакшука\\ шиша\\ ишиша\\ потшиша
\end{multicols}
\end{activity}

\section{14.02.2025}
\begin{itemize}
  \item Мише носи шешир.
  \item Бошко оди во шумата.
  \item Саше собира шишарки.
  \item Тоше го намести шаторот.
  \item Избриши ја прашината! 
  \item Мишко го избуши ѕидот.
  \item Сношти ме болеше мешето.
  \item Рашо се потшиша.
  \item На клупата седеше и пишуваше.
  \item Нешто ми шушна и се исплашив.
  \item Што боиш? 
  \item Маша се тушира.
\end{itemize}

\section{19.02.2025}
\begin{song}
  Цреши, праски, вишни, сливи\\
  миризливо ќе расцветаат!\\
  Вредни пчелки и пеперутки\\
  кај нив ќе долетраат!\\

  Штркот штрака на цел глас\\
  нова пролет, еве иде\\
  расцутено сѐ ќе биде.
\end{song}

\section{26.02.2025}
\noindent
\begin{minipage}[t]{0.45\textwidth}
  жа\\ же\\ жи\\ жо\\ жу
\end{minipage}
\hfill
\begin{minipage}[t]{0.45\textwidth}
аж\\ eж\\ иж\\ ож\\ уж
\end{minipage}

\vspace{1cm}

\noindent
\begin{minipage}[t]{0.18\textwidth}
aжа\\ aже\\ aжи\\ aжо\\ aжу
\end{minipage}
\hfill
\begin{minipage}[t]{0.18\textwidth}
ежа\\ еже\\ ежи\\ ежо\\ ежу
\end{minipage}
\hfill
\begin{minipage}[t]{0.18\textwidth}
ижа\\ иже\\ ижи\\ ижо\\ ижу
\end{minipage}
\hfill 
\begin{minipage}[t]{0.18\textwidth}
ожа\\ оже\\ ожи\\ ожо\\ ожу
\end{minipage}
\hfill 
\begin{minipage}[t]{0.18\textwidth}
ужа\\ уже\\ ужи\\ ужо\\ ужу
\end{minipage}

\vspace{1cm}

\noindent
\begin{minipage}[t]{0.18\textwidth}
жажа\\ жаже\\ жажи\\ жажо\\ жажу
\end{minipage}
\hfill
\begin{minipage}[t]{0.18\textwidth}
жежа\\ жеже\\ жежи\\ жежо\\ жежу
\end{minipage}
\hfill
\begin{minipage}[t]{0.18\textwidth}
жижа\\ жиже\\ жижи\\ жижо\\ жижу
\end{minipage}
\hfill
\begin{minipage}[t]{0.18\textwidth}
жожа\\ жоже\\ жожи\\ жожо\\ жожу
\end{minipage}
\hfill
\begin{minipage}[t]{0.18\textwidth}
жужа\\ жуже\\ жужи\\ жужо\\ жужу
\end{minipage}

\begin{instruction}
  \begin{itemize}
    \item Устата е во форма на крукче кај гласот Ш.
    \item Забите се малку одвоени.
    \item Врвот на јазикот е свртен нагоре позади горните запчиња.
  \end{itemize}
\end{instruction}


\section{28.02.2025}
\begin{minipage}[t]{0.3\textwidth}
жа жа жааа\\ жа жа жеее\\ жа жа жиии\\ жа жа жооо\\ жа жа жууу
\end{minipage}
\hfill
\begin{minipage}[t]{0.3\textwidth}
же же жааа\\ же же жеее\\ же же жиии\\ же же жооо\\ же же жууу
\end{minipage}
\hfill
\begin{minipage}[t]{0.3\textwidth}
жи жи жааа\\ жи жи жеее\\ жи жи жиии\\ жи жи жооо\\ жи жи жууу
\end{minipage}

\vspace{0.5cm}

\begin{minipage}[t]{0.45\textwidth}
жо жо жааа\\ жо жо жеее\\ жо жо жиии\\ жо жо жооо\\ жо жо жууу
\end{minipage}
\hfill
\begin{minipage}[t]{0.45\textwidth}
жу жу жааа\\ жу жу жеее\\ жу жу жиии\\ жу жу жооо\\ жу жу жууу
\end{minipage}

\section{14.03.2025}
\begin{activity}{Зборчиња}
\begin{multicols}{4}
жаба\\ жал\\ жалба\\ Жарко\\ жена\\ желба\\ жерав\\ животно\\ живот\\ желка\\ жирафа\\ жива\\ жито\\ желе\\ жица\\ Живко\\ жеден\\ жилет\\ жолчка\\ жубори\\ журнал
\end{multicols}
\end{activity}

\begin{instruction}
Вежбајте со продолжување на гласот „Ж“. Пр. \emph{жжжелка} и посилен изговор.
\end{instruction}

\section{19.03.2025}
\begin{activity}{Зборчиња}
\begin{multicols}{4}
кожа\\ кажа\\ може\\ покажи\\ лежи\\ лажица\\ гаража\\ може\\ пожар\\ плажа\\ мрежа\\ ружа\\ важи\\ ножици\\ виножито\\ фрижидер\\ грижи\\ Кожув\\ Блаже\\ ужинка\\ ужива\\ ожедне
\end{multicols}
\end{activity}

\begin{instruction}
\emph{ко-жжжа} и да ве гледа во уста за да се потсети на позиција на гласот „Ж“.
\end{instruction}

\section{21.03.2025}
\begin{activity}{Зборчиња}
\begin{multicols}{4}
нож\\ маж\\ еж\\ багаж\\ бакнеж\\ грабеж\\ колаж\\ дожд\\ гужва\\ тужба\\ важно\\ тажно\\ вежба\\ кожно\\ влажно
\end{multicols}
\end{activity}

\begin{instruction}
Повторно нагласувајте го и продолжувајте го гласот „Ж“ со посилен изговор.
\end{instruction}

\section{26.03.2025}
\begin{itemize}
  \item Живка живие во Штип.
  \item Жане има жолта блуза.
  \item Паркирај ја колата во гаража.
  \item Ми се исполни желбата.
  \item Желката е бавна.
  \item Ежот има боцки.
  \item Имам нешто важно да ти кажам.
  \item Врне дожд.
  \item Во лето одиме на плажа.
  \item За ужинка имавме банана.
  \item Жирафата има долг врат.
  \item Жарко е жеден.
  \item Се боцна со жица.
  \item Сечев колаж со ножици.
  \item Стави го шишето во фрижидер.
\end{itemize}

\section{28.03.2025}
\begin{minipage}[t]{0.45\textwidth}
  ча\\ че\\ чи\\ чо\\ чу
\end{minipage}
\hfill
\begin{minipage}[t]{0.45\textwidth}
  ач\\ еч\\ ич\\ оч\\ уч
\end{minipage}

\vspace{1cm}

\noindent
\begin{minipage}[t]{0.18\textwidth}
  ача\\ аче\\ ачи\\ ачо\\ ачу
\end{minipage}
\hfill
\begin{minipage}[t]{0.18\textwidth}
  еча\\ ече\\ ечи\\ ечо\\ ечу
\end{minipage}
\hfill
\begin{minipage}[t]{0.18\textwidth}
  ича\\ иче\\ ичи\\ ичо\\ ичу
\end{minipage}
\hfill 
\begin{minipage}[t]{0.18\textwidth}
  оча\\ оче\\ очи\\ очо\\ очу
\end{minipage}
\hfill 
\begin{minipage}[t]{0.18\textwidth}
  уча\\ уче\\ учи\\ учо\\ учу
\end{minipage}

\begin{activity}{Зборчиња}
\begin{multicols}{4}
часовник\\ чад\\ чамец\\ чавки\\ чади\\ чепка\\ чека\\ четка\\ често\\ чорба\\ чеша\\ чадор\\ чукни\\ четири\\ чизми\\ чаша\\ чисто\\ чита\\ чорапи\\ чоколадо\\ четврток\\
чаршав\\ рече\\ пече\\ мече\\ пречи\\ маче\\ плаче\\ куче\\ сече\\ влече\\ тече\\ бучава\\ прваче\\ печено\\ ручек\\ кочи\\ качи\\ печурки\\ вечера\\ значи\\ меч\\ клуч\\
пејач\\ пливач\\ водич\\ кауч\\ ично\\ квачка\\ пајче\\ момче\\ маичка\\ рачка\\ печка\\ брчи\\ вчера\\ точно\\ бавча
\end{multicols}
\end{activity}

\begin{itemize}
  \item Чедо чека автобус. 
  \item Мачето е гладно.
  \item Во бавчата има печурки.
  \item Вчера си купив нови чизми. 
  \item Често одам во Кочани.
\end{itemize}

\begin{instruction}
  Потсетувајте го устата да е напред, и посилен изговор.
\end{instruction}

\section{02.04.2025}
\begin{minipage}[t]{0.45\textwidth}
  џа\\ џе\\ џи\\ џо\\ џу
\end{minipage}
\hfill
\begin{minipage}[t]{0.45\textwidth}
  аџ\\ еџ\\ иџ\\ оџ\\ уџ
\end{minipage}

\vspace{1cm}

\begin{minipage}[t]{0.18\textwidth}
  аџа\\ аџе\\ аџи\\ аџо\\ аџу
\end{minipage}
\hfill
\begin{minipage}[t]{0.18\textwidth}
  еџа\\ еџе\\ еџи\\ еџо\\ еџу
\end{minipage}
\hfill
\begin{minipage}[t]{0.18\textwidth}
  иџа\\ иџе\\ иџи\\ иџо\\ иџу
\end{minipage}
\hfill 
\begin{minipage}[t]{0.18\textwidth}
  оџа\\ оџе\\ оџи\\ оџо\\ оџу
\end{minipage}
\hfill 
\begin{minipage}[t]{0.18\textwidth}
  уџа\\ уџе\\ уџи\\ уџо\\ уџу
\end{minipage}

\begin{activity}{Зборчиња}
\begin{multicols}{4}
џамлија\\ џеб\\ џем\\ Џули\\ џамлија\\ џемпер\\ џамбо\\ џугла\\ џип\\ Џон\\ џигер\\ џокер\\ Џони\\ џудо\\ џагор\\ џумка\\ џин\\ џез\\ пинџур\\
манџа\\ Маџари\\ баџо\\ џуџе\\ нинџа\\ маџун\\ буџет\\ гоџи\\ беџ\\ оџак\\ оџа\\ оџачар\\ џуџе\\ филџан\\ тенџере\\ канџа\\ манџа\\ џвака\\ џунџуле
\end{multicols}
\end{activity}

\begin{itemize}
  \item Со Џони изигравме со џамлии. 
  \item Во џебот имам парички. 
  \item Од Џамбо си купив играчки. 
  \item На челото имам џумка. 
  \item Манџата е вкусна. 
  \item Во филџанот има кафе. 
  \item Џули вози џип. 
  \item Мечката има канџи.
\end{itemize}

\begin{song}
  \textsl{Оџак}\\

Симо џвака\\ 
кришка со џем,\\ 
облечен во нов\\
џемпер со џеб.\\


Малку од џемот\\
на џемперот му капна\\
како беџ врз џебот\\
застана, стапна.
\end{song}

\section{04.04.2025}
\noindent
\begin{minipage}[t]{0.3\textwidth}
  са-ша\\ се-ше\\ си-ши\\ со-шо\\ су-шу
\end{minipage}
\hfill
\begin{minipage}[t]{0.3\textwidth}
  аса-аша\\ асе-аше\\ аси-аши\\ асо-ашо\\ асу-ашу
\end{minipage}
\hfill
\begin{minipage}[t]{0.3\textwidth}
  еса-еша\\ есе-еше\\ еси-еши\\ есо-ешо\\ есу-ешу
\end{minipage}

\vspace{1cm}

\noindent
\begin{minipage}[t]{0.45\textwidth}
иса-иша\\ исе-ише\\ иси-иши\\ исо-ишо\\ ису-ишу
\end{minipage}
\hfill
\begin{minipage}[t]{0.45\textwidth}
оса-оша\\ осе-оше\\ оси-оши\\ осо-ошо\\ осу-ошу
\end{minipage}

\vspace{1cm}

\noindent
\begin{minipage}[t]{0.45\textwidth}
уса-уша\\ усе-уше\\ уси-уши\\ усо-ушо\\ усу-ушу
\end{minipage}
\hfill
\begin{minipage}[t]{0.45\textwidth}
ас-аш\\ ес-еш\\ ис-иш\\ ос-ош\\ ус-уш
\end{minipage}

\begin{instruction}
Вежбајте ги истиве комбинации со З-Ж, Џ-Ч, Ѕ-Џ, без да ве гледа во уста.
\end{instruction}

\noindent
\begin{minipage}[t]{0.45\textwidth}
супа - шупа\\ сака - шака\\ коса - коша\\ чаша - часа\\ маса - маша
\end{minipage}
\hfill
\begin{minipage}[t]{0.45\textwidth}
шума - сума\\ шега - сега\\ уста - уште\\ што - сто\\ реси - реши
\end{minipage}

\vspace{1cm}

\noindent
\begin{minipage}[t]{0.45\textwidth}
чело - цело\\ раче - раце
\end{minipage}
\hfill
\begin{minipage}[t]{0.45\textwidth}
крцка - крчка\\ зрачи - зраци
\end{minipage}

\vspace{1cm}

\noindent
\begin{minipage}[t]{0.45\textwidth}
вазна - важна\\ коза - кожа\\ зелка - желка
\end{minipage}
\hfill
\begin{minipage}[t]{0.45\textwidth}
заби - жаби\\ гриза - грижа\\ мази - мажи
\end{minipage}

\section{09.04.2025}
\begin{activity}{Зборчиња}
\begin{multicols}{4}
суша\\ Сашо\\ сѐуште\\ секогаш\\ суштина\\ сошие\\ скрши\\ смешно\\ снешко\\ шанса\\ пасош\\ штос\\ шест\\ шумско\\ зачин\\ сначи\\ чисто\\ чамец
чизма\\ задача\\ железо\\ зошто\\ возеше\\ ножици\\ железница\\ жица\\ цреша\\ Суничица\\ женско
\end{multicols}
\end{activity}

\begin{itemize}
    \item Симо шета сам. 
    \item Зоки живее во Железара. 
    \item Мишо сака шумско овошје. 
    \item Шиме седи во соба. 
    \item Цане чека автобус. 
    \item Живко сака желе бонбони. 
    \item Жане црта цреши. 
    \item Џони зема кришка леб.
\end{itemize}

\begin{song}
  \textsl{Саша и Мише}\\

  Саша има машина\\
  фустанчиња шие\\
  за куклата шарена\\
  во шкафот што спие.\\


  Шие Саша шие,\\
  а братчето Мише \\
  отиде по сок \\
  со шарено шише.
\end{song}

\section{16.04.2025}
Вежби за гласот „Л“. 

\begin{instruction}
За правилен изговор на гласот „Л“
\begin{itemize}
  \item усните се во насмевка
  \item забите се малку отворени 
  \item јазичето е позади горните заби (за почеток додека добие навика дека треба да го користи јазикот може да го гризне со забите)
\end{itemize}
\end{instruction}

\noindent
ллллл - Л\\
лллл - Ла\\
лллл - Ле\\
лллл - Ли\\
лллл - Ло\\
лллл - Лу\\

Вежби за усни и јазик:

\begin{itemize}
  \item насмевка - бакнеж 
  \item јазик горе-долу позади заби со отворена уста
\end{itemize}

\section{23.04.2025}

\noindent
\begin{minipage}[t]{0.18\textwidth}
ла\\ ла\\ ла\\ ла\\ ла
\end{minipage}
\hfill
\begin{minipage}[t]{0.18\textwidth}
ле\\ ле\\ ле\\ ле\\ ле
\end{minipage}
\hfill
\begin{minipage}[t]{0.18\textwidth}
ли\\ ли\\ ли\\ ли\\ ли
\end{minipage}
\hfill 
\begin{minipage}[t]{0.18\textwidth}
ло\\ ло\\ ло\\ ло\\ ло
\end{minipage}
\hfill 
\begin{minipage}[t]{0.18\textwidth}
  лу\\ лу\\ лу\\ лу\\ лу
\end{minipage}


\section{25.04.2025}
\begin{activity}{Зборчиња}
ла, ла, ла - лапне, ладно, лав\\
ле, ле, ле - леб, лечи, лето\\
ли, ли, ли - лила, лик, лист\\
ло, ло, ло - лото, ловец, локум\\
лу, лу, лу - лути, лук, лушпа
\end{activity}

\section{07.05.2025}
\noindent
\begin{minipage}[t]{0.18\textwidth}
ала\\ але\\ али\\ ало\\ алу
\end{minipage}
\hfill
\begin{minipage}[t]{0.18\textwidth}
ела\\ еле\\ ели\\ ело\\ елу
\end{minipage}
\hfill
\begin{minipage}[t]{0.18\textwidth}
ила\\ иле\\ или\\ ило\\ илу
\end{minipage}
\hfill 
\begin{minipage}[t]{0.18\textwidth}
ола\\ оле\\ оли\\ оло\\ олу
\end{minipage}
\hfill 
\begin{minipage}[t]{0.18\textwidth}
ула\\ уле\\ ули\\ уло\\ улу
\end{minipage}

\begin{activity}{Зборчиња}
  \begin{multicols}{4}
молам\\ боли\\ бела\\ балон\\ пчела\\ колено\\ мила\\ кула\\ нула\\ весела\\ школо\\ цело\\ поле\\ кала\\ мала\\ желе\\ долу\\ балет\\ палачинки\\ чоколадо
салама\\ салата\\ малина\\ скали\\ камила\\ пали\\ пелени\\ салон\\ волан\\ село\\ недела\\ табела\\ балада\\ кошула
\end{multicols}
\end{activity}

\begin{instruction}
  Делете го зборот на слогови за полесен изговор. Пр. \emph{са-ла}, \emph{мо-лам}.
\end{instruction}

\appendix
\section{Полезные ресурсы} 
\begin{itemize}
  \item Книги по логопедии
  \item Веб-сайты с упражнениями
  \item Образцы артикуляционной гимнастики
\end{itemize}

% Document ends
\end{document}
